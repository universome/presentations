\documentclass[10pt]{beamer}

%\usepackage[backend=bibtex,firstinits=true,style=verbose-inote,citestyle=authortitle]{biblatex}
\usepackage{bm}
\usepackage{graphicx}
\usepackage{subcaption}
\usepackage{amsmath}
\usepackage{amsfonts}
\usepackage{makecell}
\usepackage{filecontents}
\usepackage{biblatex}
\newcommand{\expect}[2][]{
\ifthenelse{\equal{#1}{}}{
\mathbb{E}\left[#2\right]
}{
\underset{#1}{\mathbb{E}}\left[#2\right]
}}

\newcommand{\cov}[2][]{
\ifthenelse{\equal{#1}{}}{
\text{Cov}\left[#2\right]
}{
\underset{#1}{\text{Cov}}\left[#2\right]
}}


\newcommand{\var}[2][]{
\ifthenelse{\equal{#1}{}}{
\text{Var}[#2]
}{
\underset{#1}{\text{Var}}[#2]
}}

\newcommand{\loss}[2][]{
\ifthenelse{\equal{#1}{}}{
\mathcal{L}(#2)
}{
\mathcal{L}_{#1}(#2)
}}

\newcommand{\kl}[2]{
\text{D}_\text{KL}[#1 \parallel #2]
}

\newcommand{\R}{\mathbb{R}}
%\newcommand{\Prob}{\mathbb{P}}

\newcommand{\1}[1]{\mathds{1}\{#1\}}


%\usecolortheme{dolphin}
\setbeamertemplate{navigation symbols}{}
\setbeamertemplate{section in toc}{\inserttocsectionnumber.~\inserttocsection}

\begin{filecontents*}{references.bib}
@inproceedings{ImplementationMattersInRL,
title={Implementation Matters in Deep RL: A Case Study on PPO and TRPO},
author={Logan Engstrom and Andrew Ilyas and Shibani Santurkar and Dimitris Tsipras and Firdaus Janoos and Larry Rudolph and Aleksander Madry},
booktitle={International Conference on Learning Representations},
year={2020},
url={https://openreview.net/forum?id=r1etN1rtPB}
}
\end{filecontents*}

\addbibresource{references.bib}


\title{Implementation Matters in Deep RL: A Case Study on PPO and TRPO\footnote{\citepaper{ImplementationMattersInRL}, ICLR 2020}\footnote{Summary: \href{http://tiny.cc/zh0lnz}{http://tiny.cc/zh0lnz}}}
%\author{Ivan Skorokhodov}
\date{}
%\logo{\includegraphics[height=1cm]{images/ipavlov-logo.png}}

\newcommand{\citepaper}[1]{\citetitle{#1} by \citeauthor{#1}}

%\graphicspath{{./images}}

%\usetheme{lucid}
\begin{document}

\begin{frame}
    \titlepage
\end{frame}

\begin{frame}{Overview}
    \begin{itemize}
        \item\pause Performance of many RL algorithms depends a lot on tricks used in implementation
        \item\pause Authors dived into PPO and found out that it is not an exception
        \item\pause Key observation: tricks used in PPO give more boost than PPO itself
        \item\pause PPO+tricks works a bit better than TRPO+tricks
    \end{itemize}
\end{frame}

\begin{frame}{A reminder: off-policy PG vs TRPO vs PPO}
\pause
1. \textbf{Baseline}. A common part between TRPO and PPO is the off-policy policy gradient (with importance sampling):

\begin{equation}
J_\text{PG}(\theta) = \expect[\left(s_{t}, a_{t}\right) \sim \pi]{\frac{\pi_{\theta}\left(a_{t} | s_{t}\right)}{\pi\left(a_{t} | s_{t}\right)} \hat{A}_{\pi}\left(s_{t}, a_{t}\right)}
\end{equation}

\pause
2. \textbf{TRPO}. TRPO differs from PG by constraining the optimization step:
\begin{equation}
\begin{array}{cl}
\max _{\theta} & J_\text{PG}(\theta) \\
\text { s.t. } & D_{K L}\left(\pi_{\theta}(\cdot | s) \| \pi(\cdot | s)\right) \leq \delta
\end{array}
\end{equation}

\pause
3. \textbf{PPO}. PPO differs from PG by clipping the ratio inside the objective:
\begin{equation}
J_\text{PPO} = \expect[\left(s_{t}, a_{t}\right) \sim \pi]{\min \left(\operatorname{clip}\left(\rho_{t}, 1-\varepsilon, 1+\varepsilon\right) \hat{A}_{\pi}\left(s_{t}, a_{t}\right), \rho_{t} \hat{A}_{\pi}\left(s_{t}, a_{t}\right)\right)}
\end{equation}
where
\begin{equation}
\rho_{t}=\frac{\pi_{\theta}\left(a_{t} | s_{t}\right)}{\pi\left(a_{t} | s_{t}\right)}.
\end{equation}

\end{frame}

\begin{frame}{A ton of trick in PPO paper}
Authors looked at the PPO implementation by OpenAI and found a lot of tricks in it.
\pause The main ones were:
\begin{enumerate}
    \item\pause Value function clipping (during value function fitting)
    \item\pause Reward scaling
    \item\pause Orthogonal init + layer scaling
    \item\pause Learning rate annealing
\end{enumerate}
\end{frame}

\begin{frame}{What is the main source of a good PPO performance?}
\pause Authors ran several experiments for PPO and TRPO with and without tricks and obtained the following results

\begin{center}
\footnotesize
\begin{tabular}{l c c c}
\hline & \text { WALKER2D-V2 } & \text { HOPPER-V2 } & \text { HUMANOID-V2 } \\
\hline
\text { PG +tricks } & 2867 & 2371 & 831 \\
\hline
\text { PG +PPO } & 2735 & 2142 & 674 \\
\text { PG +TRPO } & 2791 & 2043 & 586 \\
\hline
\text { PG +PPO +tricks } & \textbf{3292} & \textbf{2513} & 806 \\
\text { PG +TRPO+tricks } & 3050 & 2466 & \textbf{1030} \\
%\hline
%\text { PPO (original version) } & 3424 & 2316 & - \\
\hline
\end{tabular}
\end{center}

\pause So,
\begin{itemize}
    \item tricks improve the performance better than PPO or TRPO
    \item original paper should have used tricks for TRPO as well
\end{itemize}

\end{frame}

\end{document}
