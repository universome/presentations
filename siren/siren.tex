\documentclass[handout, 10pt]{beamer}

%\usepackage[backend=bibtex,firstinits=true,style=verbose-inote,citestyle=authortitle]{biblatex}
\usepackage{bm}
\usepackage{graphicx}
\usepackage{subcaption}
\usepackage{amsmath}
\usepackage{amsfonts}
\usepackage{makecell}
\usepackage{filecontents}
\usepackage{biblatex}
\usepackage{xcolor}
% \input{../new-commands.tex}

%\usecolortheme{dolphin}
\setbeamertemplate{navigation symbols}{}
\setbeamertemplate{section in toc}{\inserttocsectionnumber.~\inserttocsection}

\begin{filecontents*}{references.bib}
@misc{SIREN,
    title={Implicit Neural Representations with Periodic Activation Functions},
    author={Vincent Sitzmann and Julien N. P. Martel and Alexander W. Bergman and David B. Lindell and Gordon Wetzstein},
    year={2020},
    eprint={2006.09661},
    archivePrefix={arXiv},
    primaryClass={cs.CV}
}
\end{filecontents*}

\addbibresource{references.bib}


\title{Implicit Neural Representations with Periodic Activation Functions\footnote{\citepaper{SIREN}}}
%\subtitle{}
%\author{Ivan Skorokhodov}
%\date{}
%\logo{\includegraphics[height=1cm]{images/ipavlov-logo.png}}

\newcommand{\citepaper}[1]{\citetitle{#1} by \citeauthor{#1}, \citeyear{#1}}

%\graphicspath{{./images}}

%\usetheme{lucid}
\begin{document}

\begin{frame}
    \titlepage
\end{frame}

\begin{frame}{Overview}
    \begin{itemize}
        \item\pause We can represent images/3D-objects/video/audio/etc as \textit{implicit neural representations} (INR)
        \item\pause INR is a neural network $\Phi(v)$ which takes a coordinates vector $v$ and produces a pixel/voxel/frequency/etc value at that point:
        \begin{figure}
            \centering
            \includegraphics[width=0.4\textwidth]{images/inr}
        \end{figure}
        \item\pause I.e. one neural network corresponds to a single image.
        \item\pause Authors showed that it is very helpful to use periodic activation functions
    \end{itemize}
\end{frame}

\end{document}
