\documentclass{article}
\usepackage[english, russian]{babel}
\usepackage[utf8]{inputenc}
\usepackage{fancyhdr}
\usepackage{amsmath,amssymb,amsfonts,amsthm}
\usepackage{indentfirst}
\usepackage{bm}
\usepackage[a4paper, total={6in, 10in}]{geometry}

% For using subfigures
\usepackage{graphicx}
\usepackage{caption}
\usepackage{subcaption}

% To put a figure where it is
\usepackage{float}

\usepackage{tensor}
\usepackage{biblatex}
\usepackage{titlesec}
\usepackage{bbm, dsfont} % For indicator functions
\usepackage{mathtools} % for \coloneqq
\usepackage[bottom]{footmisc} % prevents pictures going below footnote :|
% \usepackage{pifont} % for \ding{...}


% \newcommand\defeq{\stackrel{\mathclap{\normalfont\mbox{def}}}{=}}

\newcommand{\expect}[2][]{
\ifthenelse{\equal{#1}{}}{
\mathbb{E}\left[#2\right]
}{
\underset{#1}{\mathbb{E}}\left[#2\right]
}}

\newcommand{\cov}[2][]{
\ifthenelse{\equal{#1}{}}{
\text{Cov}\left[#2\right]
}{
\underset{#1}{\text{Cov}}\left[#2\right]
}}


\newcommand{\var}[2][]{
\ifthenelse{\equal{#1}{}}{
\text{Var}[#2]
}{
\underset{#1}{\text{Var}}[#2]
}}

\newcommand{\loss}[2][]{
\ifthenelse{\equal{#1}{}}{
\mathcal{L}(#2)
}{
\mathcal{L}_{#1}(#2)
}}

\newcommand{\kl}[2]{
\text{D}_\text{KL}[#1 \parallel #2]
}

\newcommand{\R}{\mathbb{R}}
%\newcommand{\Prob}{\mathbb{P}}

\newcommand{\1}[1]{\mathds{1}\{#1\}}

% We need this for \begin{theorem}...\end{theorem} staff
\newtheorem*{theorem*}{Theorem}
\newtheorem{theorem}{Theorem}
\newtheorem*{proposition*}{Proposition}
\newtheorem{proposition}{Proposition}
\newtheorem*{lemma*}{Lemma}
\newtheorem{lemma}{Lemma}


\title{Information Bottleneck in Deep Learning}
\author{Ivan Skorokhodov}

\begin{document}

%\maketitle

\section{Information Theory basics}
Here we will give all the required definitions and their properties.

\begin{definition}
\textbf{Entropy} H(X) of a discrete random variable $X$ with probability mass function $p(X)$ is the quantity
\[
H(X) = \expect[p(x)]{-\log p(x)}
\]
\end{definition}

\begin{definition}
\textbf{Differential entropy} of a random variable $X$ with probability density function $p(x)$ is the quantity
\[
H(X) = \expect[p(x)]{-\log p(x)}
\]
\end{definition}

Mutual information:
\[
\begin{split}
I(X;Y)
&= H(X) - H(X|Y) \\
&= H(Y) - H(Y|X) \\
&= \kl{p(x,y)}{p(x)p(y)} \\
&= \text{TC}(p(x,y)) \\
&= \expect[x]{\kl{p(y|x)}{p(y)}}
\end{split}
\]

Chain rule for MI:
\[
I(X ; Y, Z)=I(X ; Z) + I(X ; Y | Z)
\]

Conditional MI identities:
\[
\begin{split}
I(X ; Y|Z)
&= \expect[Z]{\kl{p(x,y|z)}{p(x|z)p(y|z)}} \\
&= H(X, Z) + H(Y, Z) - H(X, Y, Z) - H(Z) \\
&= H(X|Z) - H(X|Y, Z) \\
&= H(X|Z) + H(Y|Z) - H(X, Y|Z) \\
\end{split}
\]

Conditional entropy:
\[
H_{p}(y | z) :=\mathbb{E}_{y, z \sim p(y, z)}[-\log p(y | z)]
\]

Conditional cross-entropy:
\[
H_{p, q}(y | z) := \mathbb{E}_{y, z \sim p(y, z)}[-\log q(y | z)]
\]


\section{Information Bottleneck as such}
\subsection{Origins and formulation}
Information Bottleneck origins from rate-distortion theory, which is solving the following task: given a random variable $X$ and some \textit{distortion} function $d(x,\tilde{x})$ quantize $X$ into $\tilde{X}$ such that $\tilde{X}$ is compressed as much as possible, but not too corrupted.
Strictly speaking, we are trying to solve
\[
\min_{p(\tilde{x} | x)} I(\tilde{X}; X) \quad \text{s.t. } D(X, \tilde{X}) \leq D^*,
\]
where
\[
D(X, \tilde{X}) = \expect[p(x, \tilde{x})]{d(x, \tilde{x})}
\]
and $D^*$ is the maximum value of possible distortion that we permit.
MSE loss or Hamming distance are common choices for $d(x, \tilde{x})$.

%We solve this problem by introducing lossy coding $X \xrightarrow{f} Z \xrightarrow{g} \tilde{X}$, where $f$ and $g$ are encoder and decoder.

%Why do we need it?
%Imagine you want to send a bag of memes to your grandma.
%Your grandma lives in countryside, where internet connection is quite week.
%

%Note, that here we minimize $I(\tilde{X}; X)$ and not $H(\tilde{X})$ which can feel more natural, is equivalent for deterministic mappings and actually being done sometimes.
%The reason for this is that (in discrete case) we are maximizing $H(X|\tilde{X})$ by doing this, and this means that we try to cover more $X$ with a single $\tilde{X}$.
%This, in turn, makes $\tilde{X}$ be set in those regions where $X$ is uniform and capture equal amounts of $X$.
%While minimization of $H(\tilde{X})$ can provide a solution, where TODO.

We can find optimal quantization by solving the variational problem:
\[
\loss{p(\tilde{x} | \tilde{x})} = I(X; \tilde{X}) + \beta D(X, \tilde{X})
\]

The main problem with rate-distortion theory is that we need a distortion function $d(x, \tilde{x})$, which is difficult to specify for complex structured data, such as video or speech.
And here Information Bottleneck comes to the rescue: what if after reconstruction we are only interested in some variable $Y$?
Then we can reformulate our problem as
\[
\loss{p(\tilde{x} | \tilde{x})} = I(X; \tilde{X}) - \beta I(Y, \tilde{X})
\]

For discrete case it can be shown \cite{Information_Bottleneck}, that it's a special case of rate-distortion problem with $d(x, \tilde{x}) = \kl{p(y|x)}{p(y|\tilde{x})}$.

\subsection{A few words about sufficient statistics}
\begin{itemize}
    \item Two datasets which give us the same inference about sufficient statistic, would give the same inference about underlying parameter $\theta$.
    \item Any injective function of sufficient statistic is also a sufficient statistic.
    \item Factorization theorem says, that $p(x|y) = h_T(X) g_T(T(X), y)$, dependence in $g_T$ in $x$ is only through $T(x)$.
\end{itemize}

\subsection{Sufficiency and minimality of representations}
From now on we'll use symbol $Z$ for variable $\tilde{X}$, because it's more consistent with modern DL literature.

We have the following optimization problem:
\[
\loss{p(x;z)} = I(X;Z) - \beta I(Z; Y)
\]

There is an interesting statement which connects notion of \textit{sufficiency} and \textit{minimality} of $Z$ in the $\beta \to \infty$ setting.
Let denote by $F(X)$ all random mappings of $X$ (this means, that for $f \in F(X)$ we have Markov chain $Y \to X \to f(X)$).
Let $S_X(Y)$ be a set of all sufficient statistics of $X$ for $Y$.

\begin{proposition}
If $Z$ is a solution to
\[
\min_{Z} I(X; Z) \quad\text{s.t. } I(Z;Y) = \max_{Z'} I(Z'; Y)
\]
then $Z$ is a minimal sufficient statistics of $X$ for $Y$
\[
P(X|Z,Y) = P(X|Z)
\]
\end{proposition}

In other words, we are getting a minimal sufficient representation by optimizing Lagrangian in the limit $\beta \to \infty$.
The proof of this theorem is split into two lemmas \cite{IB_learning_and_generalization}.

\begin{lemma}
    $Z$ is a sufficient statistic of $X$ for $Y$ iff $I(Z;Y) = I(X;Y)$.
\end{lemma}

\begin{proof}
Imagine that $T \in S_X(Y)$.
For any $Z \in F(X)$ we have a Markov chain $Y \to X \to Z$, so by DPI we have $I(Y;Z) \leq I(Y;X)$.
But by sufficient statistic property we have $P(X|Z,Y) = P(X|Y)$, so we have a Markov chain $Y \to Z \to X$.
Again by DPI we have $I(Y;Z) \leq I(Y;X)$, so $I(Y;Z) = I(Y;X)$.

Now consider $Z = f(X)$ for some $f \in F(X)$ such that $I(Y;Z) = I(Y;X)$.
As $Y \to X \to Z$ is a Markov chain, then by definition of conditional MI we have:
\[
\begin{split}
I(Y:Z|X) &\triangleq \expect[X]{\kl{p(Y,Z|X)}{p(Y|X)p(Z|X)}} \\
&= \expect[X]{\kl{p(Z|X,Y)p(Y|X)}{p(Y|X)p(Z|X)}} \\
&= \expect[X]{\kl{p(Y|X)p(Z|X)}{p(Y|X)p(Z|X)}} = 0
\end{split}
\]

Now, by chain rule for MI we have:
\[
I(Y : X,Z) = I(Y : X) + I(Y,X|Z) = I(Y : X) + I(Y,Z|X) \Longrightarrow I(Y,X|Z) = I(Y,Z|X) = 0
\]
Applying definition of conditional MI again we get $p(Y,X|Z) = p(Y|Z)p(X|Z)$.
And this means that $Z \in S_X(Y)$:
\[
p(X|Y,Z) = \frac{p(X,Y|Z)}{p(Y|Z)} = \frac{p(X|Z) p(Y|Z)}{p(Y|Z)} = p(X|Z)
\]
\end{proof}

Let's denote by $S_X^*(Y)$ a set of minimal sufficient statistics of $X$ for $Y$.
\begin{lemma}
Let $Z \in S_X(Y)$, then
\[
Z \in S_X^*(Y) \Longleftrightarrow I(X;Z) = \min_{T \in S_X(Y)} I(X;T)
\]
\end{lemma}

\begin{proof}
First, let $Z$ be a minimal sufficient statistic.
Then for any other sufficient statistic $T$ we have $Z = f(T)$ for some $f$.
Then we get a Markov chain $X \to T \to Z$ and by DPI we have $I(X;Z) \leq I(X;T)$.

Now, let's prove reverse direction of the claim.
Imagine, that $Z \in S_X(Y)$ but is not minimal.
We are going to show, that $\exists T \in S_X(Y)$ such that $I(X;Z) > I(X;T)$.
By Fisher-Neyman factorization theorem we have
\[
Z \in S_X(Y) \Longleftrightarrow \exists\ h_Z, g_Z \text{ s.t. } \forall x,y \quad p(x|y) = h_Z(x) g_Z(Z(x), y)
\]

Let's define an equivalence relation
\[
a \sim b \Longleftrightarrow \forall y \exists \lambda s.t. \frac{g_Z(a, y)}{g_Z(b, y)} = \lambda(a, b)
\]

Now we define a deterministic function $T:\mathcal{X} \to \mathcal{Z}$ such that $\forall x: T(x) = \bar{z}$ --- a representative of $[Z(x)]$ (TODO: exists by axiom of choice?).
%This means that $T$ is a function of $Z$ and we have a Markov chain $X \to Z \to T$.
Let's prove, that it is a sufficient statistic.
For this let's define
\[
\begin{split}
h_T(x) &\triangleq h_Z(x) \frac{g_Z(Z(x), y)}{g_Z(T(x), y)} \\
g_T(T(x), y) &\triangleq g_Z(T(x), y)
\end{split}
\]
Then we have
\[
p(x|y) = h_Z(x) g_Z(Z(x), y) = h_Z(x) \frac{g_Z(Z(x), y)}{g_T(T(x), y)} g_T(T(x), y) = h_T(x) g_T(T(x), y)
\]
Hence $T$ is a sufficient statistic.

Now let's show that $I(X;Z) > I(X;T)$.
Since $Z$ is not minimal, then there is such $R \in S_X(Y)$ that $Z$ is not a function of $R$.
Let's show, that $T$ is a function of $R$ (btw, this will show, that $T$ is minimal).
For this we are going to show that if $R(x_1) = R(x_2)$ then $T(x_1) = T(x_2)$: this would allow us to build a function $\phi:\mathcal{R} \to \mathcal{T}$ which just take value $r$, find it's preimage $R^{-1}(r)$, take any sample $x \in R^{-1}(r)$ and compute $T(x)$.
For any $x_1, x_2$ such that $R(x_1) = R(x_2)$ we have:
\[
\begin{split}
\frac{g_Z(Z(x_1), y)}{g_Z(Z(x_2), y)}
&= \frac{p(x_1 | y) h_Z(x_2)}{p(x_2 | y) h_Z(x_1)} \\
&= \frac{h_R(x_1) g_R(R(x_1), y) h_Z(x_2)}{h_R(x_2) g_R(R(x_2), y) h_Z(x_1)} \\
&= \frac{h_R(x_1) g_R(R(x_1), y) h_Z(x_2)}{h_R(x_2) g_R(R(x_1), y) h_Z(x_1)} \\
&= \frac{h_R(x_1) h_Z(x_2)}{h_R(x_2) h_Z(x_1)} \\
&= \lambda(Z(x_1), Z(x_2))
\end{split}
\]
This means that $Z(x_1) \sim Z(x_2)$, which in turn means that $T(x_1) = T(x_2)$, so $T$ is minimal sufficient statistic and a function of $R$.
\end{proof}

\section{First steps: SZT experiments, critics}

\section{Disentangled representations}

The following theorem gives us a hope to build invariant and disentangled representations.
Informally, it says that if we can build sufficient and minimal representations (which we are building by optimizing IB Lagrangian, for example), then we get invariant and disentangled ones.

\begin{proposition}
If $\eta$ is a nuisance for the task $y$ and $z$ is a sufficient representation of $x$ and we have a Markov chain $\eta \to x \to z$, then
\[
I(z;\eta) \leq I(z;x) - I(x;y)
\]

Moreover, if $y$ is discrete, then
\end{proposition}

\begin{proof}
As we have a Markov chain $(y,\eta) \to x \to z$ then by DPI $I(z;y,\eta) \leq I(z;x)$.
By chain rule we have
\[
I(z; \eta) = I(z; y, \eta) - I(z; y | \eta) \leq I(z;x) - I(z; y | \eta)
\]

By definition of nuisance $y \perp \eta$ so $I(z;y|\eta) \geq I(z;y)$, because (by one of identities for conditional MI):
\[
\begin{split}
I(z;y|\eta)
&= H(y|\eta) - H(y|z,\eta) \\
&= H(y) - H(y|z,\eta) \\
&\geq H(y) - H(y|z) \\
&= I(z;y)
\end{split}
\]

As $z$ is sufficient, i.e. $I(x;y) = I(z;y)$ we obtain
\[
I(z;\eta) \leq I(z;x) - I(z; y | \eta) \leq I(z;x) - I(z;y) = I(z;x) - I(x;y)
\]

Now consider, that we have $p(x;y)$ and $y$ is discrete.
Then by task-nuisance decomposition lemma we can introduce a nuisance $\eta$ s.t. $x = f(y;\eta)$ and $f$ is deterministic.
That's why we have
\[
I(z;x) = I(z; y, \eta) = I(z;\eta) + I(z; y | \eta)
\]
Rearranging terms we get
\[
I(z;\eta) = I(z;x) - I(x;y|\eta) = I(z;x) - I(x;y) - \underbrace{(I(x;y|\eta) - I(x;y))}_\varepsilon
\]

Let's prove that $0 \leq \varepsilon \leq H(y|x)$.
By definition of $\varepsilon$:
\begin{align*}
\varepsilon &= I(z;y|\eta) - I(x;y)
\shortintertext{Since $y|\eta \to x|\eta \to z|\eta$ is a Markov chain, then by DPI:}
&\leq I(x;y|\eta) - I(x;y) \\
&= H(y|\eta) - H(y|x,\eta) - H(y) + H(y|x) \\
&= H(y) - H(y|x,\eta) - H(y) + H(y|x) \\
&= H(y|x) - H(y|x,\eta) \\
%\shortintertext{Since $y$ is discrete and $H(y|x,\eta) \geq 0$:}
%\shortintertext{TODO: actually, to measure information-theoretic things between discrete and continuous r.v. we should convert discrete to delta-distribution, so $H(y|x,\eta)$ can be negative.}
\shortintertext{The last inequality is subtle, actually. We use the fact $H(y|x) - H(y|x,\eta) = I(y; \eta, x) \geq 0 \Rightarrow H(y|x) \geq H(y|x,\eta)$. And $\epsilon \geq 0$ follows from equation proved above: $I(z;y|\eta) \geq I(z;y) = I(x;y)$. That's why $H(y|x) \geq 0$, because it's a valid upper bound.}
&\leq H(y|x)
\end{align*}

\end{proof}

What does it give us? Does it give us invariance and disentanglement?
First of all, we see, that as more minimal our $z$ the more invariant it is, because minimal sufficient statistic implies that $I(x;z)$ is the most minimal possible!
And the first term is the only term we can influence with our $z$.

Unfortunately, it does not give us disentanglement yet.
To ensure disentanglement we need additional assumptions.
But before diving into it, let's think about how can we achieve invariance with what we already now?

\begin{corollary}
Minimizing IB Lagrangian with $\beta \to 0$
\[
H(y|z) + \beta I(x;z)
\]
\end{corollary}

\begin{proof}
This is true by theorem proved above.
\end{proof}

Introducing bottlenecks by adding noise or reducing dimensionality.
TODO: shouldn't we prove that bottlenecks reduces MI?
\begin{corollary}
Imagine we have a Markov chain $(y,\eta) \to x \to z_1 \to z_2$ and $I(z_1;z_2) < I(x; z_1)$, i.e. there is some bottleneck on the road between $z_1 \to z_2$. Then if $z_2$ is sufficient, it is more invariant    
\end{corollary}

\begin{proof}[Bottlenecks promote invaraince]
\begin{align*}
I(z_2;\eta) &\leq I(z_2;z_1) - I(z_1,y) = I(z_2; z_1) - I(x, y) < I(z_1; x) - I(x,y)
\end{align*}
\end{proof}

\begin{corollary}
Imagine setup above.
If $y$ is discrete and is a deterministic function of $x$, then inequality is strict.
\end{corollary}

\begin{proof}
We have
\[
\begin{split}
I(z_2; \eta) \leq I(z_2; z_1) - I(x;y) < I(z_1; x) - I(x;y) = I(z_1; \eta) + \varepsilon
\end{split}
\]
Since $y$ is a deterministic function of $x$ then $\varepsilon = 0$, hence the desired result.
\end{proof}

\begin{corollary}[Stacking increases invariance]
Consider we a have a Markov chain of layers
\[
(y,\eta) \to x \to z_1 \to ... \to z_l
\]
If $z_L$ is still sufficient, then it's more invariant.
\end{corollary}

TODO:
Well, this is because we have a Markov chain $\eta \to ... \to z_l$, it's not a corollary:
$I(\eta; z_l) \leq I(\eta; z_i)$.

\section{Information in the weights}
\subsection{Properties}
\subsection{Minimum description length}
\subsection{Connection to representations}

\section{Mutual information estimators}
\subsection{MINE}

\subsection{MI estimator under gaussian convolutions}

\section{Bibliography}
\printbibliography

\section*{Exercises}
\begin{enumerate}
%    \item Prove that
%\[
%\arg \min_{q(y)} \kl{p(x,y)}{p(x)q(y)} = f(y) \Longleftrightarrow f(y) = p(y)
%\]
    \item Prove that
\[
\arg \min_{q_1(x_1) , ... , q_n(x_n)} \kl{p(x_1, ..., x_n)}{q_1(x_1) \cdot ... \cdot q_n(x_n)} = p(x_1) \cdot ... \cdot p(x_n) \Longleftrightarrow q_1(x_1) , ... , q_n(x_n) = p(x_1), ..., p(x_n)
\]
%    \item We have a natural (which comes from our modeling) Markov property of $X \to Z \to Y$. Prove Markov property of $Z \to X \to Y$.
    \item Prove that if $X, Y$ are normally distributed, then $Z$ is normally distributed too. %http://www.jmlr.org/papers/volume6/chechik05a/chechik05a.pdf
    \item Maybe something on finding differential entropy of r.v. with infinite differential entropy. Or constructing such a variable.
    \item Prove that $I(X;T)$ is either infinity or constant (except measure zero of weights).
    \item Prove DPI
    \item IB solution is a minimal sufficient statistic (based on http://www.cs.huji.ac.il/labs/learning/Papers/ibgen.pdf, theorem 5)
\end{enumerate}

\end{document}
